\subsection{Model \#1: fixed stars, fixed planets, twin binaries}
\label{sec:model_1}

Consider a universe in which all planets are identical, and all stars are 
either single or twin stars with otherwise identical physical properties.
The occurrence rate density at a planet radius $r$, stellar radius $R$, and 
semimajor axis $a$ for this model is
\begin{equation}
\Gamma(r,R,a) = \sum_i w_i \Lambda_i \delta^3(r_p,R_\star,a_p),
\label{eq:model1_occ_rate_density}
\end{equation}
for $r_p$ and $R_\star$ some fixed planet and stellar radii, $a_p$ a fixed 
semi-major axis, and $\delta$ the Dirac delta function, whose compact
form will be used for brevity.
The occurrence rate over any interval that includes $r_p$, $R_\star$, and 
$a_p$ is
\begin{equation}
\Lambda = \sum_i w_i \Lambda_i = \frac{ \sum_i N_i \Lambda_i }{N_{\rm tot}}.
\end{equation}
The rate is zero over intervals that do not.

To express the rate density of detected planets, $\hat{\Gamma} = \sum 
Q_i\Gamma_i$, we need the detection efficiencies for each system type, which 
are products of the geometric and selection probabilities:
\begin{align}
Q_i(\vec{x}) &= Q_{g,i}(\vec{x}) Q_{c,i}(\vec{x}),\quad {\rm where}\ 
\vec{x}=(r,R,a).
\label{eq:general_detection_efficiency}
\end{align}
Similar to Pepper et al. (2003), but in a new context, we take $Q_c$ as the 
ratio of the number of stars that were searchable to the number of stars that 
were selected.
Assuming a homogeneous distribution of stars, this gives
\begin{equation}
Q_{c,i}(\vec{x}) = \left(
\frac{d_{{\rm det},i}(\vec{x})}{d_{\rm sel}(\vec{x})}
\right)^3,
\end{equation}
for $d_{\rm sel}$ the maximum distance to which surveyed stars are selected, 
and $d_{{\rm det},i}$ the maximum distance to which planets can actually be 
detected about the $i^{\rm th}$ system type.
Note that $d_{\rm sel} \geq d_{{\rm det},i}$.
In a signal-to-noise limited transit survey in which the observer does not 
know which stars are binaries, 
\begin{equation}
d_{\rm sel} \propto (r/R)^2 (L_{\rm sys} T_{\rm dur} A N_{\rm tra})^{1/2},
\end{equation}
for $L_{\rm sys}=L_1(1+\ell)$ the system luminosity, $T_{\rm dur}$ the 
transit duration, $A$ the detector area, and $N_{\rm tra}$ 
the number of observed transits.
However,
\begin{equation}
d_{{\rm det},i} \propto \mathcal{D}_i(r/R)^2 (L_{\rm sys} T_{\rm dur} A N_{\rm 
tra})^{1/2},
\label{eq:d_det_i}
\end{equation}
for the dilution $\mathcal{D}_i$ given by
\begin{align}
\mathcal{D}_i
&=
\left.
\begin{cases}
1 & \text{for } i=0,\ {\rm single} \\
L_1 / L_{\rm sys} = (1+\ell)^{-1}, & \text{for } i=1,\ {\rm primary} \\
\ell L_1 / L_{\rm sys} = (1 + \ell^{-1})^{-1}, & 
    \text{for } i=2,\ {\rm secondary},
\end{cases}
\right.
\label{eq:dilution}
\end{align}
where the light ratio $\ell$ of a given binary is defined as the ratio of 
the luminosity of the secondary to the primary.

The maximum detectable distance to single stars is assumed to be known, and so 
$d_{{\rm sel},0} = d_{{\rm det},0}$.
For binary systems there is a necessary incompleteness, and combining 
Eqs.~\ref{eq:general_detection_efficiency} through~\ref{eq:dilution} yields
\begin{align}
Q_0 &= Q_{g,0}Q_{c,0} = Q_{g,0} \label{eq:detection_efficiency_0}\\
Q_1 &= Q_{g,1}Q_{c,1} = Q_{g,0} (1+q^\alpha)^{-3} \\
Q_2 &= Q_{g,2}Q_{c,2} = Q_{g,0} q^{2/3} (1+q^{-\alpha})^{-3} q^{-5}, 
\label{eq:detection_efficiency_2}
\end{align}
for $Q_{g,0}=R/a$, the transit probability in single star systems.
The factors of $q^{2/3}$ and $q^{-5}$ in
Eq.~\ref{eq:detection_efficiency_2} come from the assumed stellar 
mass-luminosity-radius relation: $R\propto M \propto L^{1/\alpha}$.
For $q=1$ both terms evaluate to unity, but they will later become relevant.

Summarizing, we have written the rate density for each system type
(Eq.~\ref{eq:model1_occ_rate_density}) and the detection efficiency for each 
system type 
(Eq.~\ref{eq:detection_efficiency_0}-\ref{eq:detection_efficiency_2}),
and so have fully specified the rate density of detected planets, 
in addition to the true rate density.


\paragraph{What does an observer ignoring binarity infer?} 
An observer who ignores binarity assumes a detection efficiency 
$\tilde{Q}=Q_0$,
measures a detected planet rate density $\tilde{\Gamma}$, 
and infers an apparent rate density $\Gamma_a$.
Analogous to Eq.~\ref{eq:detected_rate_density},
\begin{equation}
\tilde{\Gamma} = \Gamma_a \tilde{Q}.
\end{equation}
Accounting for dilution, one can show
\begin{equation}
\Gamma_a = 
w_a \Lambda_0 \delta^3(r_p, R_\star, a_p) +
w_b (\Lambda_1 Q_{c,1} + \Lambda_2 Q_{c,2}) 
				\delta^3(r_p/\sqrt{2}, R_\star, a_p),
\end{equation}
for $w_a = N_0/(N_0+N_1)$, and $w_b = N_1/(N_0+N_1)$.
This observer miscounts the number of total searched stars, does not correct 
for incompleteness, and misclassifies the planetary radii because of dilution.

\paragraph{Correction to inferred rate density and inferred rate}

Define a rate density correction factor, $X_\Gamma$, as the ratio of the 
apparent to true rate densities:
\begin{equation}
X_\Gamma \equiv \frac{\Gamma_a}{\Gamma}.
\end{equation}
This factor can be a function of whatever parameters $\Gamma_a$ and $\Gamma$ 
depend on; in this study, the planet radius is most relevant.
For the twin-binaries model,
\begin{equation}
X_\Gamma(r)
=
\frac{w_a \Lambda_0\delta^3(r_p) + 
	w_b(\Lambda_1 Q_{c,1} + \Lambda_2 Q_{c,2}) \delta^3(r_p/\sqrt{2})  }
	{(w_0\Lambda_0 + w_1\Lambda_1 + w_2\Lambda_2)\delta^3(r_p)}
	\label{eq:model1_correction}
	\end{equation}
where $\delta^3(r_p)$ is shorthand for $\delta^3(r-r_p,R-R_\star,a-a_p)$.

If we take the rates $\Lambda_i$ to be equal, applying the definitions of 
the weights gives a rate density correction factor at $r=r_p$ of
$X_\Gamma(r_p) = (1+\mu)^{-1}$, where 
\begin{align}
\mu \equiv \frac{N_1}{N_0} &=
\frac{n_b}{n_s} \left(\frac{d_{\rm sel,b}}{d_{\rm sel,s}}\right)^3 = 
\frac{{\rm BF}}{1-{\rm BF}} (1+\ell)^{3/2},
\label{eq:mu_definition}
\end{align}
for $n_b$ and $n_s$ the number density of binaries and singles in a 
volume limited sample.
Using Raghavan et al. (2010)'s $0.7-1.3M_\odot$ multiplicity fraction as our 
binary fraction\footnote{
The binary fraction is the fraction of systems in a volume-limited sample that 
are binary. It is equivalent to the multiplicity fraction if there are no 
triple, quadruple, or higher order multiples.
}, we set ${\rm BF}=0.44$.
The resulting correction to the rate density is $X_\Gamma(r_p) \approx 0.31$. 
The correction at $r_p/\sqrt{2}$ is infinite.
The numerical realization of this model agrees with these analytic values, and 
its output is shown in Fig~\ref{fig:errcases_model_1}.
%beta = 2.2223355980148636
If instead we assume that $\Lambda_0 = \Lambda_1$, but that $\Lambda_2=0$, we 
find 
$X_\Gamma(r_p) = (1+2\mu)/(1+\mu)^2$.
Taking the same binary fraction, this evaluates to $X_\Gamma(r_p)\approx 0.53$.
Since the correction to the rate is equal to that of rate density, at 
$r=r_p$, the occurrence rate is underestimated by a factor of roughly 2 to 3.

%Note that a correction to the inferred rate, $X_\Lambda$, can be 
%defined analogously:
%\begin{equation}
%X_\Lambda \equiv \frac{\Lambda_a}{\Lambda}.
%\end{equation}
%For this twin binary model, the correction to the rate is the same as that to 
%the rate density.
