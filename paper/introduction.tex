\section{Introduction}

A group of astronomers wants to measure the mean number of planets of a 
certain type per star of a certain type.
Ignoring stellar multiplicity, they perform a signal-to-noise limited transit 
survey and detect $N_{\rm det}$ signals that seem to be from the desired 
class of planet.
They count the apparent number of selected stars, $N_\star$.
Assuming their survey is limited by Poisson noise, our astronomers compute 
their detection efficiency $p_{\rm det}$ as a function of planetary and 
stellar properties, in order to know which of the selected stars were
searchable.
Correcting for the geometric transit probability $p_{\rm tra}$, they report an 
apparent occurrence rate $\Lambda_{\rm a}$:
\begin{equation}
\Lambda_{\rm a} = \frac{N_{\rm det}}{N_\star} \times \frac{1}{p_{\rm tra}}
								  \times \frac{1}{p_{\rm det}}.
\label{eq:wouldnt_it_be_nice}
\end{equation}

There are many potential pitfalls.  Some genuine transit signals can be missed
by the detection pipeline.  Some apparent transit signals are spurious, from
noise fluctuations, failures of `detrending', or instrumental effects.  Stars
and planets can be misclassified due to statistical and systematic errors in
the measurements of their properties.  Poor angular resolution causes false
positives due to blends with background eclipsing binaries. {\it Et cetera}.

Here we focus on problems that arise from the fact that many stars exist in 
multiple star systems.
For simplicity, we consider only binaries, and we assume that they are all 
spatially unresolved.

An immediate complication is that, due to dynamical stability or some 
aspect of planet formation, the occurrence rate of planets might differ 
between binary and single-star systems.
If ``occurrence rate'' is defined as the mean number of planets within 
set radius and period bounds per star in a given mass interval, it must 
implicitly marginalize over stellar multiplicity.
This means marginalizing over ``occurrence rates in single star systems'', 
``occurrence rates about primaries'', and
``occurrence rates about secondaries'' (see {\it e.g.,} Wang et al., 
2015).

Outside of astrophysical differences, there are observational biases for every 
term in Eq.~\ref{eq:wouldnt_it_be_nice}.
There are errors in $N_{\rm det}$ due to planet radius misclassification.
There are errors in $N_\star$ because a given selected star might in fact be 
two stars.
There are errors in $p_{\rm tra}$ because stars in binaries may have different 
masses and radii than assumed in the single-star case.
Finally, there are errors in $p_{\rm det}$ because the detection efficiencies 
of planets that orbit single stars, primaries, and secondaries are all 
different.

Correcting for binarity's observational biases is non-trivial.
For instance, in counting the number of selected stars, even after realizing 
that binaries count as two stars, one must note that the multiplicity 
fraction of {\it selected} stars is greater than that of a volume limited 
sample.
This is the familiar Malmquist bias: binaries are selected out to larger 
distances than single stars because they are more luminous.
As a separate challenge, finding the correct number of detected planets in a 
radius bin, $N_{\rm det}$, requires knowledge of the true planetary radii.
Observers deduce apparent radii.
In binaries, the apparent and true radii differ because of diluting flux, 
and possibly because the planet is assumed to orbit the wrong star ({\it 
e.g.}, Furlan et al. 2017).
%Finally, the detection efficiency $p_{\rm det}$ is different for selected 
%stars in binary systems than it is in single systems.

To gain intuition for the many observational biases at play,
we study the following idealized transit surveys:
\begin{itemize}
    \item Model \#1: fixed stars, fixed planets, twin binaries;
    \item Model \#2: fixed planets and primaries, varying secondaries;
    \item Model \#3: fixed primaries, varying planets and secondaries.
\end{itemize}
In Sec.~\ref{sec:numerical_methods}, we introduce our numerical approach 
to the problem, and in Sec.~\ref{sec:analytic_preliminaries} we clarify our 
terminology.
We present the analytic and numerical results in 
Secs.~\ref{sec:model_1}-\ref{sec:model_3}, where each subsection corresponds 
to each model above.
We interpret these calculations throughout, and in 
Sec.~\ref{sec:discussion} discuss their relevance to topical questions in 
the interpretation of transit survey occurrence rates.
In particular, we mention the ``hot Jupiter rate discrepancy'', the relevance 
towards measurements of $\eta_\oplus$, and the significance for the dearth of 
planets recently discovered by Fulton et al. (2017).
