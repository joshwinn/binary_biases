\section{Introduction}

An astronomer who does not believe in stellar multiplicity wants to measure the
mean number of planets of a certain type per star of a certain type.
They perform a signal-to-noise limited transit survey and detect $N_{\rm det}$ 
transit signals that appear to be from planets of the desired type.  They 
calculate the apparent number of searchable stars, $N_\star$.  ``Searchable 
stars'' are the stars for which 
planets are observed with 100\% detection efficiency. 
Correcting for the geometric 
transit probability $f_{\rm g}$, they compute an apparent occurrence rate 
$\Lambda_{\rm a}$:

\begin{equation}
\Lambda_{\rm a} = \frac{N_{\rm det}}{N_\star} \times \frac{1}{f_g}.
\end{equation}

There are many potential pitfalls.  Some genuine transit signals can be missed
by the detection pipeline.  Some apparent transit signals are spurious, from
noise fluctuations, failures of `detrending', or instrumental effects.  Stars
and planets can be misclassified due to statistical and systematic errors in
the measurements of their properties.  Poor angular resolution causes false
positives due to blends with background eclipsing binaries. {\it Et cetera}.

Here we focus on problems that arise from the fact that many stars exist in 
multiple star systems.
For simplicity, we only consider binaries, and we assume that they are all 
spatially unresolved.

An immediate complication is that, due to dynamical stability or some 
aspect of planet formation, the occurrence rate of planets may differ 
between binary and single-star systems.
If ``occurrence rate'' is defined purely as the mean number of planets within 
set radius and period bounds per star in a given mass interval, it must 
implicitly marginalize over stellar multiplicity.
This means marginalizing over ``occurrence rates in single star systems'', 
``occurrence rates about primaries'', and
``occurrence rates about secondaries'' (see {\it e.g.,} Wang et al., 
2015).

Outside of astrophysical differences, there are observational biases.
A given apparently-searchable star may truly be a single star of the desired 
type. If it is binary, it may contain two, one, or no searchable stars of the 
desired type, which affects $N_\star$.
There may be systematic errors in stellar parameter estimates of such 
systems, which in this work we neglect.
One might also imagine an apparently-searchable star which in no stellar 
component is of the desired type. We will not consider errors of this type.  
We will assume that all the apparently-searchable stars are either single 
stars of the desired type, or binaries in which either the primary or else 
both components are of the desired type.

When counting the detected signals from planets of the desired 
type, $N_{\rm det}$, ignoring binarity leads to errors in planet radius 
classification, and the assumed survey completeness.
Detected planets in binary systems will have underestimated radii because of 
the diluting flux from the companions, and possibly because they are assumed 
to orbit the wrong star ({\it e.g.}, Furlan et al. 2017).
The number of selected stars should change dependent on the planet radius 
and period for which the rate measurement is desired, and on the survey's 
achieved photometric precision.
This means our naive astronomer assumes the fraction of their selected stars 
which are searchable is 1 (this is their ``assumed completeness'').
Binarity confuses this, because the ability to search for planets 
in binary systems depends on which stellar component hosts the planet, and on 
how much dilution affects the measurement.

\paragraph{How Has Binarity Been Considered in Occurrence Rate Measurements?}
Many authors have reported planet occurrence rates from transit 
survey data\footnote{
    An online list of occurrence rate papers is maintained at 
    \url{https://exoplanetarchive.ipac.caltech.edu/docs/occurrence_rate_papers.html}
}.
With regard to binarity, the typical approach has been to ignore the 
issue\footnote{Deacon et al., 2015 are an exception to the rule.}.
For instance, in a laudable study by Burke et al. (2015), they
\begin{quote}
``treat the pipeline completeness as having no uncertainty 
due to the incomplete understanding of the stellar parameters, eccentricity, 
and stellar binarity.''
\end{quote}
Though they do ``explore sensitivity in the derived planet occurrence rates 
to alternative assumptions for the stellar parameters and non-zero 
eccentricities'', they do not for binarity.

Of course, on a system-by-system level stellar multiplicity affects the 
interpretation of planet candidates. High resolution imaging 
campaigns have consequently been undertaken to measure the multiplicity of 
almost all {\it Kepler}\ Objects of Interest 
(Howell et al. 2011; Adams et al. 2012, 2013; Horch et al. 2012, 2014; 
Lillo-Box et al. 2012, 2014; Dressing et al. 2014; Law et al. 2014; Cartier et 
al. 2015; Everett et al. 2015; Gilliland et al. 2015; Wang et al. 2015a, 
2015b; Baranec et al. 2016).
The results of these programs have been collected by Furlan et al. 
(2017), and they represent an important advance in understanding the KOI 
sample's multiplicity statistics.
In particular, they can be immediately applied to rectify binarity's influence 
on the mass-radius diagram (Furlan \& Howell 2017).

For {\it Kepler}-derived occurrence rates, the high resolution imaging 
campaigns have not yet come full circle to observe a comparison sample of 
non-KOI host stars.
The most recent rate studies have thus used Furlan et al. (2017)'s 
catalog to test the effects of removing KOI hosts with known companions, which 
is an important step towards reducing contamination in the ``numerator'' of 
the occurrence rate (Fulton et al. 2017).
However, without an understanding of the multiplicity statistics of the 
non-KOI hosting stars that are assumed to be searchable, the 
true completeness, the true number of searchable stars, and thus the true 
occurrence rates will remain uncertain.


\paragraph{The Hot Jupiter Rate Discrepancy}
There is at least one context in which measurement of absolute 
occurrence rates may already be showing the signatures of binarity.
Hot Jupiter occurrence rates measured by transit surveys ($\approx 0.5\%$) are 
marginally lower than those found by radial velocity surveys ($\approx 1\%$; 
see Table~\ref{tab:hj_rates}).
Though the discrepancy is of weak statistical significance ($<3\sigma$),
one plausible explanation for the difference is that the populations have 
distinct metallicities.
As originally argued by Gould et al. (2006), the RV sample is biased towards 
metal-rich stars, which have been measured by RV surveys to preferentially 
host more giant planets (Santos et al 2004, Fischer and Valenti 2005).
The {\it Kepler}\ sample specifically has been measured to be more metal poor 
than the local neighborhood, with a mean metallicity of $[{\rm M/H}]_{\rm 
mean}\approx -0.05$ (Dong et al., 2014; Guo et al., 2017).
Studying the problem in detail, Guo et al. recently argued that the 
metallicity difference could account for a $\approx 0.1\%$ difference in the 
measured rates between the CKS and {\it Kepler}\ samples~--~not a $\approx 
0.5\%$ difference.
Guo et al. concluded that ``other factors, such as binary contamination and 
imperfect stellar properties'' must also be at play.

Aside from surveying stars of varying metallicities, radial velocity and 
transit surveys differ in how they treat binarity.
Radial velocity surveys typically reject both visual and spectroscopic binaries
({\it e.g.}, Wright et al. 2012).
Transit surveys typically observe binaries, but the question of whether they 
were searchable to begin with is usually left for later interpretation.
In spectroscopic follow-up of candidate transiting planets, the prevalence of 
astrophysical false-positives may also lead to a bias against confirmation of 
transiting planets in binary systems.

Ignoring these complications, in this work we focus on whether
binarity might intrinsically bias transit survey occurrence rates, simply 
through its effects on the number of searchable stars and the apparent radii 
of detected planets.

To progressively gain intuition, we study the following idealized transit 
surveys:
\begin{itemize}
    \item Model \#1: fixed stars, fixed planets, twin binaries;
    \item Model \#2: fixed planets and primaries, varying secondaries;
    \item Model \#3: fixed primaries, varying planets and secondaries.
\end{itemize}
In Sec.~\ref{sec:numerical_methods}, we introduce our numerical approach 
to the problem, and in Sec.~\ref{sec:analytic_preliminaries} we clarify our 
terminology.
We present the analytic and numerical results in 
Secs.~\ref{sec:model_1}-\ref{sec:model_3}, where each subsection corresponds 
to each model above.
We interpret these calculations throughout, and in 
Sec.~\ref{sec:discussion} discuss their relevance to various questions in 
the interpretation of transit survey occurrence rates. 
