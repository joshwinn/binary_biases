\section{Discussion}
\label{sec:discussion}

\paragraph{How Has Binarity Been Considered in Occurrence Rate Measurements?}
Many authors have computed planet occurrence rates using transit 
survey data\footnote{
    An online list of occurrence rate papers is maintained at 
    \url{https://exoplanetarchive.ipac.caltech.edu/docs/occurrence_rate_papers.html}
}.
Notable studies using {\it Kepler}\ data include those by Howard et al. 2012, 
Fressin et al., 2013, Foreman-Mackey et al., 2014, Dressing \& Charbonneau 
2015, and Burke et al. 2015.
Mostly if not entirely, these studies have ignored stellar multiplicity.
However, binarity clearly introduces a level of systematic uncertainty to 
pipeline completeness, as well as to star and planet counts.
No one has yet carefully studied the magnitude of this issue for 
{\it Kepler} occurrence rates.
While the present work does not resolve this problem, it suggests the 
approximate order of magnitude for the necessary correction factors.
%NOTE: the above might be better in a "discussion" section?

Of course, on a system-by-system level stellar multiplicity affects the 
interpretation of planet candidates. High resolution imaging 
campaigns have been undertaken to measure the multiplicity of 
almost all {\it Kepler}\ Objects of Interest 
(Howell et al. 2011; Adams et al. 2012, 2013; Horch et al. 2012, 2014; 
Lillo-Box et al. 2012, 2014; Dressing et al. 2014; Law et al. 2014; Cartier et 
al. 2015; Everett et al. 2015; Gilliland et al. 2015; Wang et al. 2015a, 
2015b; Baranec et al. 2016).
The results of these programs have been collected by Furlan et al. 
(2017), and they represent an important advance in understanding the KOI 
sample's multiplicity statistics.
In particular, they can be immediately applied to rectify binarity's influence 
on the mass-radius diagram (Furlan \& Howell 2017).

%TODO: talk about N. R. Deacon et al 2015, Pan-STARRS1 here

To help clarify {\it Kepler}-derived occurrence rates, the high resolution 
imaging campaigns have not yet come full circle by observing a comparison 
sample of non-KOI host stars.
The most recent rate studies have thus used Furlan et al. (2017)'s 
catalog to test the effects of removing KOI hosts with known companions, which 
is an important step towards reducing contamination in the ``numerator'' of 
the occurrence rate (Fulton et al. 2017).
However, without an understanding of the multiplicity statistics of the 
non-KOI hosting stars that are assumed to be searchable, the 
true completeness, the true number of searchable stars, and thus the true 
occurrence rates will remain uncertain.


\paragraph{The Hot Jupiter Rate Discrepancy}
There is at least one context in which measurement of 
occurrence rates may already be showing the signatures of binarity.
Hot Jupiter occurrence rates measured by transit surveys ($\approx 0.5\%$) are 
marginally lower than those found by radial velocity surveys ($\approx 1\%$; 
see Table~\ref{tab:hj_rates}).
The discrepancy has weak statistical significance ($<3\sigma$).
That said, one reason to expect a difference is that the corresponding stellar 
populations have distinct metallicities.
As argued by Gould et al. (2006), the RV sample is biased towards 
metal-rich stars, which have been measured by RV surveys to preferentially 
host more giant planets (Santos et al 2004, Fischer and Valenti 2005).
The {\it Kepler}\ sample specifically has been measured to be more metal poor 
than the local neighborhood, with a mean metallicity of $[{\rm M/H}]_{\rm 
    mean}\approx -0.05$ (Dong et al., 2014; Guo et al., 2017).
Studying the problem in detail, Guo et al. recently argued that the 
metallicity difference could account for a $\approx 0.1\%$ difference in the 
measured rates between the CKS and {\it Kepler}\ samples~--~not a $\approx 
0.5\%$ difference.
Guo et al. concluded that ``other factors, such as binary contamination and 
imperfect stellar properties'' must also be at play.

Aside from surveying stars of varying metallicities, radial velocity and 
transit surveys differ in how they treat binarity.
Radial velocity surveys typically reject both visual and spectroscopic binaries
({\it e.g.}, Wright et al. 2012).
Transit surveys typically observe binaries, but the question of whether they 
were searchable to begin with is usually left for later interpretation.
In spectroscopic follow-up of candidate transiting planets, the prevalence of 
astrophysical false-positives may also lead to a bias against confirmation of 
transiting planets in binary systems.

Ignoring these complications, in this work we showed that
binarity does bias transit survey occurrence rates, simply through its 
effects on the number of searchable stars and the apparent radii of detected 
planets.
Specifically, our results from Sec.~\ref{sec:model_3} indicate that binarity 
could lead to underestimated HJ rates by a multiplicative factor of $\approx 
1.3$.

We assess the effect this might have towards resolving the hot Jupiter rate 
discrepancy by asking:
what is the probability of Wright et al. (2012)'s result, given a rate drawn 
from the stated bounds of Petigura et al. (in prep)? (See 
Table~\ref{tab:hj_rates}).
In other words, we first take the true HJ rate as $\Lambda_{\rm HJ} = 5.7 \pm 
1.3$, with Gaussian uncertainties, and then drawing from a Poisson 
distribution compute the probability of detecting at least 10 hot Jupiters in 
a sample of 836 stars.
Without accounting for binarity or metallicity, only 4\% of RV surveys would 
detect at least 10 hot Jupiters.
If we multiply $\Lambda_{\rm HJ}$ by $1.2$ to account for Guo et al. 
(2017)'s measured metallicity difference between the {\it Kepler}\ field and 
the local solar neighborhood, 9\% of RV surveys would detect at least 10 hot 
Jupiters.
If we then multiply again by $1.3$ to account for binarity's bias, we find that
23\% of RV surveys would detect at least 10 hot Jupiters, and any discrepancy 
would be quite tenuous
We emphasize that this result is only suggestive~--~a true resolution of the 
rate discrepancy would likely require a detailed understanding of the {\it 
Kepler}\ field's multiplicity statistics.



\paragraph{Does a detected planet orbit the primary or secondary?}
Ciardi et al. (2015) studied the effects of stellar multiplicity on the 
planet radii derived from transit surveys.
They modeled the problem for {\it Kepler}\ objects of interest by matching a 
population of binary and tertiary companions to KOI stars, 
under the assumption that the KIC-listed stars were the primaries.
They then computed planet radius correction factors assuming that {\it 
Kepler}-detected planets orbited the primary or companion stars
with equal probability (their Sec. 5).
Under these assumptions, they found that any given planet's radius is on 
average underestimated by a multiplicative factor of 1.5.

Our models show that assuming a detected planet has equal probability of 
orbiting the primary or secondary leads to an overstatement of
binarity's population-level effects.
A planet orbiting the secondary does lead to extreme corrections, but these 
cases are rare outliers, because the searchable volume for secondaries is so 
much smaller than that for primaries.
Phrased in terms of the completeness, in our Model \#3 only $\sim 6\%$ of 
selected secondaries are searchable, compared to $\sim 60\%$ of selected 
primaries.
This means that when high-resolution imaging discovers a binary companion in 
system that hosts a detected transiting planet, the planet is much
more likely to orbit the primary.
This statement is independent of the fact that planets are often confirmed to 
orbit the primary by inferring the stellar density from the transit duration.


\paragraph{On the utility for future occurrence rate measurements}
Though they will be difficult to distinguish from false positives, {\it TESS}\ 
is expected to discover over $10^4$ giant planets (Sullivan et al. 2015).
One possible use of this overwhelmingly large sample will to measure an
occurrence rate of short-period giant planets.
Our work indicates that if this measurement is to be more precise than $\sim 
30\%$, binarity cannot be neglected.


\paragraph{What about {\it Kepler}?}
Barclay \& Collaborator (in preparation) have performed the exercise 
of taking stars selected by the {\it Kepler}\ team, pairing them with a 
population of secondaries, injecting a realistic distribution of planet radii, 
and then comparing the inferred occurrence rates with the true ones.
In their model, they find that binarity leads to an inferred rate of 
Earth-sized planets $\approx 10\%$ less than the true rate.
In our Model \#3, if all $\Lambda_i$'s are equal (a plausible assumption in 
the lack of evidence to the contrary), the underestimate is by 16\%.
