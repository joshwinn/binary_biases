\section{Discussion}
\label{sec:discussion}

\paragraph{Does a detected planet orbit the primary or secondary?}
Ciardi et al. (2015) studied the effects of stellar multiplicity on the 
planet radii derived from transit surveys.
They modeled the problem for {\it Kepler}\ objects of interest by matching a 
population of binary and tertiary companions to KOI stars, 
under the assumption that the KIC-listed stars were the primaries.
They then computed planet radius correction factors assuming that {\it 
Kepler}-detected planets orbited the primary or companion stars
with equal probability (their Sec. 5).
Under these assumptions, they found that any given planet's radius is on 
average underestimated by a multiplicative factor of 1.5.

Our models show that assuming a detected planet has equal probability of 
orbiting the primary or secondary leads to an overstatement of
binarity's population-level effects.
A planet orbiting the secondary does lead to extreme corrections, but these 
cases are rare outliers, because the searchable volume for secondaries is so 
much smaller than that for primaries.
Phrased in terms of the completeness, in our Model \#3 only $\sim 6\%$ of 
selected secondaries are searchable, compared to $\sim 60\%$ of selected 
primaries.
This means that when high-resolution imaging discovers a binary companion in 
system that hosts a detected transiting planet, the planet is much
more likely to orbit the primary.
This statement is independent of the fact that planets are often confirmed to 
orbit the primary by inferring the stellar density from the transit duration.


\paragraph{On the utility for future occurrence rate measurements}
Though they will be difficult to distinguish from false positives, {\it TESS}\ 
is expected to discover over $10^4$ giant planets (Sullivan et al. 2015).
One possible use of this overwhelmingly large sample will to measure an
occurrence rate of short-period giant planets.
Our work indicates that if this measurement is to be more precise than $\sim 
30\%$, binarity cannot be neglected.


\paragraph{What about {\it Kepler}?}
Barclay \& Collaborator (in preparation) have performed the exercise 
of taking stars selected by the {\it Kepler}\ team, pairing them with a 
population of secondaries, injecting a realistic distribution of planet radii, 
and then comparing the inferred occurrence rates with the true ones.
In their model, they find that binarity leads to an inferred rate of 
Earth-sized planets $\approx 10\%$ less than the true rate.
In our Model \#3, if all $\Lambda_i$'s are equal (a plausible assumption in 
the lack of evidence to the contrary), the underestimate is by 16\%.
