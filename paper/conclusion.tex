\section{Conclusion}
\label{sec:conclusion}

This study presented three simple models for the effects of binarity on 
occurrence rates measured by transit surveys.
The most realistic of these models (Model \#3) suggests that binarity does 
lead to underestimates in transit survey occurrence rates, but with less than 
$30\%$ relative error.
The model further suggests that hot Jupiter rates measured by transit surveys 
are biased to infer $\approx 1.3\times$ fewer hot Jupiters per star than 
surveys that only measure occurrence rates about single stars ({\it 
i.e..,} radial velocity surveys).
It also indicates that binarity's effects on the measured 
occurrence rates of Earth-sized planets are far smaller than current 
systematic uncertainties.
Though our models are simplistic, their agreement with Barclay \& 
Collaborator's recent detailed simulations indicate that they may capture the 
essential ingredients.
