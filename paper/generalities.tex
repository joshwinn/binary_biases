\subsection{Analytic Preliminaries}
\label{sec:analytic_preliminaries}

Define the occurrence rate density, $\Gamma$,
as the expected number of planets per star per natural logarithmic bin 
of planetary and stellar phase space:
\begin{equation}
\Gamma(\vec{x}) = \frac{d^n\Lambda}{\prod_{i=1}^{n} {\rm d} \ln x_i  }.
\label{eq:rate_density_defn}
\end{equation}
$\vec{x}$ is an $n$-dimensional list of the continuous physical properties 
that might affect the occurrence rate density. 
For example,
$\vec{x}=(r,P,R)$ where $r$ is the planet radius, $P$ is its orbital period, 
and $R$ is the host star radius.
The occurrence rate $\Lambda$ is found by integrating the rate density over a 
specified volume of phase space (e.g., Youdin 2011).


The previous definition implicitly marginalizes the rate density over stellar 
multiplicity.
For simplicity, this work only considers single and binary star systems.
Then for a selected population of stars and planets the rate density can be 
written\footnote{
    Eq.~\ref{eq:rate_density_marginalized} follows by writing the $i^{\rm 
        th}$ system type's rate density as some normalization multiplied by a 
    probability density:
    $\Gamma_i(\vec{x}) = \mathcal{Z}_i p_i(\vec{x})$.
    For Eq.~\ref{eq:rate_density_defn} to hold, we must have $\mathcal{Z}_i = 
    \Lambda_i$.
}
\begin{equation}
\Gamma(\vec{x})
= \sum_{i=0}^{2} w_i \Gamma_i(\vec{x})
= \sum_i w_i \Lambda_i p_i(\vec{x})
\label{eq:rate_density_marginalized}
\end{equation}
where $i=0$ corresponds to single star systems, $i=1$ primaries of binaries, 
and $i=2$ secondaries of binaries.
$\Lambda_i$ is the occurrence rate integrated over all possible phase space  
for the $i^{\rm th}$ system type, and $p_i(\vec{x})$ is the joint 
probability density function so that $\Gamma_i(\vec{x}) = \Lambda_i 
p_i(\vec{x})$.
The weights are given by 
\begin{equation}
w_i = N_i/N_{\rm tot},
\end{equation}
for $N_{\rm tot} = \sum_i N_i$ the total number of selected stars, and 
$N_0,N_1,N_2$ the number of selected single stars, primaries, and 
secondaries respectively.
Both $N_1$ and $N_2$ are equal to the number of selected points on 
the sky which are in fact binary systems.
The relationship between the rate $\Lambda$ over a desired volume of 
phase space $\Omega_{\rm desired}$ and $\Lambda_i$ is
\begin{equation}
\Lambda = \sum_i
\left(
w_i \Lambda_i \int_{\Omega_{\rm desired}} p_i(\vec{x}) \,{\rm d}\Omega
\right),
\label{eq:occ_rate}
\end{equation}
where $p_i(\vec{x})$ is assumed to be normalized to unity.

A transit survey will have a rate density of detected planets $\hat{\Gamma}$, 
which for each system type will be the product of the rate density and the 
detection efficiency $Q_i(\vec{x})$:
\begin{equation}
\hat{\Gamma}(\vec{x}) = \sum_i Q_i(\vec{x}) \Gamma_i(\vec{x}) 
\equiv \sum_i \hat{\Gamma}_i(\vec{x}),
\label{eq:detected_rate_density}
\end{equation}
where again the index $i$ is over each type of system (singles, primaries, 
and secondaries).
The detection efficiency includes the geometric transit probability, as well 
as any incompleteness effects.
Foreman-Mackey et al. (2014) discuss how this 
is calculated in practice.

