%% \begin{deluxetable}{} command tell LaTeX how many columns
%% there are and how to align them.
\begin{deluxetable}{cccc}
    
%% Keep a portrait orientation

%% Over-ride the default font size
%% Use Default (12pt)

%% Use \tablewidth{?pt} to over-ride the default table width.
%% If you are unhappy with the default look at the end of the
%% *.log file to see what the default was set at before adjusting
%% this value.

%% This is the title of the table.
\caption{Occurrence rates of hot Jupiters (HJs) about FGK dwarfs, as measured 
by radial velocity and transit surveys.}
\label{tab:hj_rates}

%% This command over-rides LaTeX's natural table count
%% and replaces it with this number.  LaTeX will increment 
%% all other tables after this table based on this number
\tablenum{1}

%% The \tablehead gives provides the column headers.  It
%% is currently set up so that the column labels are on the
%% top line and the units surrounded by ()s are in the 
%% bottom line.  You may add more header information by writing
%% another line between these lines. For each column that requries
%% extra information be sure to include a \colhead{text} command
%% and remember to end any extra lines with \\ and include the 
%% correct number of &s.
\tablehead{\colhead{Reference} & \colhead{HJs per thousand stars} & 
\colhead{HJ Definition} 
%& 
%\colhead{} \\ 
%    \colhead{} & \colhead{(planets per thousand stars)} & \colhead{} & 
%    \colhead{}
} 

%% All data must appear between the \startdata and \enddata commands
\startdata
Marcy+ 2005 & 12$\pm$1 & $a<0.1\,{\rm AU}; P\lesssim10\,{\rm day}$ \\
Cumming+ 2008 & 15$\pm$6 & -- \\
Mayor+ 2011 & 8.9$\pm$3.6 & -- \\
Wright+ 2012 & 12.0$\pm$3.8 & -- \\
Gould+ 2006 & $3.1^{+4.3}_{-1.8}$ & $P<5\,{\rm day}$ \\
Bayliss+ 2011 & $10^{+27}_{-8}$ & $P<10\,{\rm day}$ \\
Howard+ 2012 & 4$\pm$1 & $P<10\,{\rm day}; r_p=8-32r_\oplus$; solar 
subset\tablenotemark{a} \\
-- & 5$\pm$1 & solar subset extended to $Kp<16$ \\
-- & 7.6$\pm$1.3 & solar subset extended to $r_p>5.6r_\oplus$. \\
Petigura+ 2017 & XX$\pm$YY & ? \\
\enddata

%% Include any \tablenotetext{key}{text}, \tablerefs{ref list},
%% or \tablecomments{text} between the \enddata and 
%% \end{deluxetable} commands

%% General table comment marker
\tablecomments{
    The upper four results are from radial velocity surveys; the latter six 
    are from transit surveys. Many of these surveys selected different 
    stellar samples.
}
\tablenotetext{a}{
    Howard+ 2012's ``solar subset'' was defined as {\it Kepler}-observed stars 
    with $4100\,{\rm K}<T_{\rm eff}<6100\,{\rm K}$, $Kp <15$, $4.0 < \log g < 
    4.9$. Their rate selected planets with measured signal to noise $>10$.
    }
    
\end{deluxetable}
